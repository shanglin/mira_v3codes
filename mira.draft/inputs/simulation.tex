\section{Simulation of M33 observations} \label{sec.simulation}

% 1. lmc mira to M33

% 2. constant + gaussian noise from sigma-mag relation

% 3. constant + under-estimated gaussian noise 

% 4. constant + gaussian noise from sigma-mag relation + few obnormal

% points (bad pixels, cosmic rays, blending, numeric failure, mismatch,
% image boundary)

% 5. eclipsing, nova, SRV, irregular variable

To build models for searching Miras from the M33 observations, we simulated Mira and Non-Mira light curves as training data. The simulated light curve includes Miras, constant stars, other type of variable stars, and constant stars with few abnormal observations. Upon the simulated light curves, we calculated the variability index and kept those with $J>0.75$. The number of light curves was chosen to match real observations, other properties such as sampling patterns and measurement uncertainties were also derived from the M33 observations. Below we describe the simulations of different kind of light curves.

\subsection{Mira light curves}
We simulated 5000 Mira light curves by applying the M33 sampling patterns and noise levels to LMC Mira light curves from the OGLE observations. The detailed description of simulation method can be found in \red{(Citation Coming Soon)}, and here we briefly describe the method we used. 

We used the model from \red{(Citation Coming Soon)} to generate template Mira light curves. The magnitudes of the templates were shifted by 6.2 mag to account for the distance difference between LMC and M33. To simulate a Mira light curve, we randomly selected a sampling pattern from the M33 observations and shifted by some amount to match the baseline of OGLE observations, then selected a template randomly but with weightss from completeness functions of M33 observations, which are derived empirically for three ranges of color.

The noise levels were derived for individual images of M33 observations. We fit the relation between magnitude uncertainties $\sigma$ and magnitude $m$ with the following empirical function
\begin{equation}
\sigma(t_i,f_k) = a(t_i,f_k)^{[m-b(t_i,f_k)]} + c(t_i,f_k)\,, \label{equ.sigma.mag}
\end{equation}
where $a$, $b$, and $c$ are free parameters and varies with time $t_i$ and field $f_k$. We then calculated the uncertainties for simulated Mira light curves based on these relations, and added Gaussian noise to their magnitudes with standard deviations at $\sigma(t_i,f_k)$.

\subsection{Constant star light curves}
We simulated  $5\times 10^5$ light curves for constant stars. The number of light curves matches the number of measured stars in the M33 observations. The method is similar to the simulation of Mira light curves, except for using constant magnitudes as templates. The magnitude distribution of the templates followed the luminosity function of the M33 observations. The sampling patterns were randomly selected from the real light curves of M33 observations, and the noise levels were calculated from Equation~\ref{equ.sigma.mag}.