We simulated 5,000 Mira light curves for as part of the training data. The simulation method is similar to that in \red{(cite GP)}, which applies the sampling patterns and noise levels of the M33 observations to LMC Mira light curves from the OGLE observations. The samping patterns we used in this paper is silightly different from those in \red{(cite GP)} because we removed some noisy frames of the M33 observations from analysis. Here we briefly describe the method we used to simulate Mira light curves.

We generated templated LMC Mira light curves based on the OGLE observations, and derived their occurrence in the M33 observations. The LMC Miras can be considered as a complete sample and are well observed, with a baseline longer than 7.5 years and hundreds of observations at relatively small uncertainty. However, the light curves are by no means continuous, which are needed for simulating sparsely observed light curves at given sampling patterns. To obtain continuous Mira templates, we used the \red{(model name)} to fit the LMC Mira light curves and predict their magnitude at any given time inside their time domain. We then shifted the template light curves by $+6.2$ magnitude to account for the distance difference between LMC and M33. For the M33 observations, most faint Miras can not be dectected. We used the completeness functions of the M33 observations derived in \red{(cite GP)} as occurrence rate of the template light curves.

The sampling patterns were randomly selected from the light curves of M33 observations. We shifted the observation dates of the templates by some random amount to cover the selected M33 sampling dates. The random shifts made the simulated light curves different from each other, even when the same template and same sampling pattern were used.

The magnitude uncertainties of the M33 observations are quite different from the LMC light curves due to differences in overall appearent magnitude and in configurations of telescopes. To simulate the noise level of the M33 observations, we fit the relations between magnitude uncertainty $\sigma$ and appearent magnitude $m$ with the following empirical function for each frame of the M33 observations
\begin{equation}
\sigma(t_i,f_k) = a(t_i,f_k)^{[m-b(t_i,f_k)]} + c(t_i,f_k)\,, \label{equ.sigma.mag}
\end{equation}
where $a$, $b$, and $c$ are free parameters and varies with frames identified by time $t_i$ and field $f_k$. The magnitude uncertainties of simulated light curves were determined by Equation~\ref{equ.sigma.mag} and normally distributed noises with standard deviations $\sigma(t_i,f_k)$ were added to the simulated light curves.