We simulated 5,000 Mira light curves for as part of the training data. The simulation method is similar to that in \red{(cite GP)}, which applies the sampling patterns and noise levels of the M33 observations to LMC Mira light curves from the OGLE observations. The samping patterns we used in this paper is silightly different from those in \red{(cite GP)} because we removed some noisy frames of the M33 observations from analysis. Here we briefly describe the method we used to simulate Mira light curves.

We generated templated LMC Mira light curves based on the OGLE observations, and derived their occurrence in the M33 observations. The LMC Miras can be considered as a complete sample and are well observed, with a baseline longer than 7.5 years and hundreds of observations at relatively small uncertainty. However, the light curves are by no means continuous, which are needed for simulating sparsely observed light curves at given sampling patterns. To obtain continuous Mira templates, we used the \red{(model name)} to fit the LMC Mira light curves and predict their magnitude at any given time inside their time domain. We then shifted the template light curves by $+6.2$ magnitude to account for the distance difference between LMC and M33. For the M33 observations, most faint Miras can not be dectected. We used the completeness functions of the M33 observations derived in \red{(cite GP)} as occurrence rate of the template light curves.

