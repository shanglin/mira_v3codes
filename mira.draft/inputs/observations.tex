\section{Observations and Photometry of M33} \label{sec.observations}

The detailed descriptions of observations and photometry procedures can be found in the first paper of the series \citep{2011ApJS..193...26P}. Here we briefly summarize them.

Almost the entire disk of M33, which spans approximately half square degree in the sky \citep{1895MNRAS..56...70R}, was observed by the DIRECT project and follow-up observations. The original DIRECT project used the Fred L. Whipple Observatory 1.2m telescope and the Michigan-Darmouth-MIT 1.3m telescope to observe M33 during 1996 September to 1999 Novemeber. Addtional observations was obtained at the Wisconsin-Indiana-Yale-NOAO Observatory 3.5m telescope between 2002 August to 2006 August. In this paper we refer them collectively as M33 observations. We only used the $I$-band data to search for Miras. The number of $I$-band observations for individual objects is shown in Figure~\ref{fig.n_obs}. It is common that there is more than one observations in one night, and as a result the median number of nights with observations is around 30. The observations consists of 29 fields, and different fields have different distributions of number of observations.

\begin{figure}
\epsscale{1}
\plotone{figures.include/edited_n_obs.eps}
\caption{The distribution of number of $I$-band observations for the objects in M33. Objects without detection in certain frames do not count as observations. The blue arrow indicates the median number of observations.}\label{fig.n_obs}
\end{figure}

We used the photometry products from \cite{2011ApJS..193...26P}. They performed PSF photometry using the DAOPHOT, ALLSTAR, and ALLFRAME \citep{1987PASP...99..191S,1994PASP..106..250S}, and carried out photometric calibration based on \cite{2006AJ....131.2478M}. We used the $I$-band light curves with Stetson's variability index $J>0.75$ \citep{1996PASP..108..851S}.
